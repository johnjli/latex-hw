\documentclass{article}

\usepackage[a4paper, total={6in, 8in}]{geometry}
\usepackage[utf8]{inputenc}
\usepackage{amsmath}
\usepackage{amssymb}
\usepackage{amsthm}
\usepackage{fancyhdr}
\usepackage{enumitem}
\usepackage[makeroom]{cancel}

\pagestyle{fancy}
\fancyhf{}
\lhead{John J Li}
\rhead{MAT300 Spring 2021 Homework 6}
\rfoot{\thepage}
\lfoot{April 2021}
\renewcommand{\headrulewidth}{0.4pt}
\renewcommand{\footrulewidth}{0.4pt}

\setlength{\parskip}{1em}
\setlength\parindent{0px}
\title{MAT300 Spring 2021 Homework 6}
\date{\today}
\author{John J Li}

\begin{document}
    \maketitle
    \thispagestyle{empty}
    \noindent\rule{\textwidth}{0.8pt}

    %###################################################################################

    \section*{Problem 1}

    Let $S = \{(1,2),(2,1),(3,3),(1,3),(3,2)\}$ and $R=\{(1,2),(2,2),(3,1),(1,3),(3,3)\}$
    be relations on $\{1,2,3\}$. Find:

    (a) $S\circ R^{-1}$

    \begin{enumerate}[label={\bfseries Solution:}, leftmargin=*]
        \item $\{(2,2),(2,3),(2,1),(1,3),(1,2),(3,2),(3,3),(3,3),(3,2)\}$
    \end{enumerate}

    (b) $R\circ (R\circ R)$

    \begin{enumerate}[label={\bfseries Solution:}, leftmargin=*]
        \item $\{(1,2),(2,2),(3,2),(3,1),(3,3),(1,2),(1,3),(1,1),(1,3),(3,2),(3,3)\}$
    \end{enumerate}

    %###################################################################################

    \section*{Problem 2}

    Let $R,S$ be relations on $A$. Show:

    (a) If $R\subseteq S$, then $R^{-1}\subseteq S^{-1}$

    \begin{proof}
        Let $R\subseteq S$ and suppose there is an arbitrary ordered pair $(x,y)\in R$ and $x,y\in A$.
        By definition of the subset, this means that there exists a $(x,y)\in S$ such that
        $(x,y)\in R = (x,y)\in S$.
        Inverting the ordered pairs would give $(y,x)\in R^{-1}$ and $(y,x)\in S^{-1}$ which
        shows $R^{-1}\subseteq S^{-1}$.
    \end{proof}

    (b) $(R\cup S)^{-1} = R^{-1}\cup S^{-1}$

    \begin{proof}
        1) Suppose an arbitrary ordered pair $(x,y)\in (R\cup S)^{-1}$ and $x,y\in A$. Then 
        $(y,x)\in R\cup S$ and so $(y,x)\in R \lor (y,x)\in S$. The inverse would be 
        $(x,y)\in R^{-1} \lor (x,y)\in S^{-1}$ thus $(x,y)\in R^{-1}\cup S^{-1}$.

        2) Suppose an arbitrary order pair $(x,y)\in R^{-1}\cup S^{-1}$ and so 
        $(x,y)\in R^{-1}\lor (x,y)\in S^{-1}$ which can be rewritten as $(y,x)\in R \lor (y,x)\in S$.
        So that $(y,x)\in (R\cup S)$; inverting the ordered pair shows $(x,y)\in (R\cup S)^{-1}$.
    \end{proof}

    %###################################################################################

    \section*{Problem 3}

    Let $R$ be a relation on $A$. Show that if $R$ is transitive and symmetric,
    then so is $R^{-1}$:

    \begin{proof}
        Suppose $R$ is transitive and symmetric, then 
        $\forall_{x\in A}\forall_{y\in A}((x,y)\in R \rightarrow (y,x)\in R)$ and
        $\forall_{x\in A}\forall_{y\in A}\forall_{\mathbb{Z}\in A}(((x,y)\in R \land (y,z)\in R) \rightarrow (x,z)\in R)$.

        Then the inverse would be 
        $\forall_{x\in A}\forall_{y\in A}((y,x)\in R^{-1} \rightarrow (x,y)\in R^{-1})$ and
        $\forall_{x\in A}\forall_{y\in A}\forall_{\mathbb{Z}\in A}(((y,x)\in R^{-1} \land (z, y)\in R) \rightarrow (z,x)\in R^{-1})$.
        So it is seen that $R^{-1}$ is also transitive and symmetric.
    \end{proof}

    %###################################################################################

    \section*{Problem 4}

    Let $R$ be the following relation on $\mathbb{Z}$, $aRb$ if $a-b\leq 10$. Check if $R$ is
    reflexive, symmetric, antisymmetric, and transitive. 

    (a) R is reflexive.
    
    \quad Let $a\in \mathbb{Z}$. Then $a-a = 0\leq 10$. So $aRa$.

    (b) R is not symmetric. 

    \quad Let $a,b\in \mathbb{Z}$. And suppose $aRb$ then there exists $b - a > 10$. 
    Let $(1,12) = (a,b)$ then $1-12 = -11 \leq 10$ but $12 - 1 = 11 \nleq 10$.

    (c) R is not antisymmetric.
    
    \quad Let $a,b\in \mathbb{Z}$ and suppose $(aRb\; \land\; bRa)$ and consider $(a,b)=(1,2)$ then
    indeed $1-2 \leq 10\; \land\; 2-1 \leq 10$ but $1\neq 2$.

    (d) R is not transitive.

    \quad Let $a,b,c\in \mathbb{Z}$ and suppose $(aRb\;\land\; bRc)$. Consider $a = 2, b = 1, c = -9$,
    then indeed $2-1\leq 10\;\land\; 1-(-9)\leq 10$ but $2-(-9) = 11 \nleq 10$.


    %###################################################################################

    \section*{Problem 5}

    Let $R$ be a relation on $\mathbb{Z}\times \mathbb{Z}$ defined as $(a,b)R(c,d)$ when $2|(a^2+c)$ and
    $3|(b-d)$. Check if $R$ is reflexive, symmetric, antisymmetric, and transitive.

    (a) R is reflexive
    \begin{enumerate}[label=\quad\quad, leftmargin=*]
        \item 
        Let $(a,b)\in \mathbb{Z}\times \mathbb{Z}$. Then $b-b = 0\cdot 3; 3|(b-b)$. And proving $2|(a^2+a)$:

        Case 1: $a$ is even 

        \quad Let $a = 2j$ where $j\in \mathbb{Z}$. Then $a^2+a = 4j^2+2j = 2(2j^2+j); 2|(a^2+a)$.

        Case 2: $a$ is odd 

        \quad Let $a = 2j +1$ where $j\in \mathbb{Z}$. Then 
        \begin{align*}
            a^2+a &= (2j+1)^2+2j+1 \\
            & = (4j^2+4j+1) + 2j + 1 \\
            & = 4j^2+6j+2 \\
            & = 2(2j^2+3j+1)
        \end{align*}

        So $2|(a^2+a)$ and $3|(b-b)$ as required.
    \end{enumerate}

    (b) R is symmetric 

    \begin{enumerate}[label=\quad\quad, leftmargin=*]
        \item 
        Let $(a,b),(c,d)\in \mathbb{Z}\times \mathbb{Z}$. And suppose $a^2+c=2k$ and $b-d=3j$ where $k,j\in \mathbb{Z}$.
        Then $d-b=3(-j)$ and in proving $2|(c^2+a)$, let $a=\sqrt{2k-c}$:

        Case 1: $c$ is even

        \quad Let $c=2i$ where $i\in \mathbb{Z}$. Then:
        \begin{align*}
            c^2+a & = c^2+\sqrt{2k-c} \\
            & = c^4 + 2k -c \\
            & = 16i^4 + 2k - 2i \\
            & = 2(8i^4+k-i)
        \end{align*}

        Case 2: $c$ is odd

        \quad Let $c=2i+1$ where $i\in \mathbb{Z}$. Then:
        \begin{align*}
            c^2+a & = c^2+\sqrt{2k-c} \\
            & = c^4 + 2k -c \\
            & = 16i^4+32i^3+24i^2+8i+1+2k-2i-1 \\
            & = 16i^4+32i^3+24i^2+6i+2k \\
            & = 2(8i^4+16i^3+12i^2+3i+k)
        \end{align*}

        So $3|(d-b)$ and $2|(c^2+a)$ as required.
    \end{enumerate}

    (c) R is not antisymmetric 

    \begin{enumerate}[label=\quad\quad, leftmargin=*]
        \item 
        Let $(a,b),(c,d)\in \mathbb{Z}\times \mathbb{Z}$. And consider $(a,b)=(1,5)$ and $(c,d)=(3,2)$, then 
        it is indeed true that $(1,5)R(1,2)$ and $(1,2)R(1,5)$ but $(3,2)\neq (1,5)$.
    \end{enumerate}

    (d) R is transitive

    \begin{enumerate}[label=\quad\quad, leftmargin=*]
        \item 
        Let $(a,b),(c,d),(e,f)\in \mathbb{Z}\times \mathbb{Z}$. And suppose $(a,b)R(c,d)$ and $(c,d)R(e,f)$,
        then $2|(a^2+c)$ and $3|(b-d)$ and $2|(c^2+e)$ and $3|(d-f)$.
        \item
        Then $b-d=3j$ and $d-f=3k$ for some $j,k\in \mathbb{Z}$. Thus $b-d+d-f=3j+3k$, then
        $b-f=3(j+k)$, so $3|(b-f)$ as required.
        \item
        Then $a^2+c=2g$ and $c^2+e=2i$ for some $g,i\in \mathbb{Z}$. 
        Then $c=\sqrt{2i-e}$ and so $a^2+\sqrt{2i-e}=2g$, which is the same as 
        $a^4+2i-e=4g^2$.
        For $2|(a^2+c)$, both 
        $a$ and $c$ have to be both even or both odd, the same is true for $2|(c^2+e)$. 
        And since $a^4$ will be odd when $a$ is odd and even when $a$ is even, 
        $a^4+2i-e=4g^2$ can be written as $a^2-e=2(2g^2-i)$ or $2|a^2-e$ as required.
    \end{enumerate}

    %###################################################################################

    \section*{Problem 6}

    Let $R$ be a relation on $\mathbb{Z}^+\times \mathbb{Z}^+$ defined as follows $(x,y)R(z,w)$ if $x|z$
    and $y-w\geq 0$.

    (a) Prove that $R$ is a partial order but not a total order.

    \begin{proof}
        $R$ is reflexive.
        \begin{enumerate}[label=\quad\quad, leftmargin=*]
            \item 
            Let $(x,y),(z,w)\in\mathbb{Z}^+\times\mathbb{Z}^+$. Then $x = 1\cdot x$ so $x|x$ and $y-y = 0\geq 0$.
            So $(x,y)R(x,y)$ as required.
        \end{enumerate}
        $R$ is transitive.
        \begin{enumerate}[label=\quad\quad, leftmargin=*]
            \item 
            Let $(x,y),(z,w),(a,b)\in\mathbb{Z}^+\times\mathbb{Z}^+$. And suppose $(x,y)R(z,w)$
            and $(z,w)R(a,b)$. So that 
            \begin{align*}
                &x|z \\
                &y-w\geq 0\\
                &z|a\\
                &w-b\geq 0
            \end{align*}
            It is seen that $y-w+w-b\geq 0$. Which can be written as $y-b\geq 0$ which satisfies one
            half of the relation.
            \item 
            Consider $x|z = x\cdot k = z$ and $z|a = z\cdot j = a$ for some $k,j\in\mathbb{Z}^+$.
            Thus $a=x\cdot k\cdot j$ and so $x|a$.
            \item
            So $(x,y)R(a,b)$ as required.
        \end{enumerate}
        $R$ is antisymmetric 
        \begin{enumerate}[label=\quad\quad, leftmargin=*]
            \item 
            Let $(x,y),(z,w)\in\mathbb{Z}^+\times\mathbb{Z}^+$ and suppose $(x,y)R(z,w)$
            and $(z,w)R(x,y)$. Thus $x\cdot k = z$ and $z\cdot j=x$ for some $k,j\in\mathbb{Z}^+$
            and $y-w\geq 0$ and $w-y\geq 0$.
            \item
            Then $z=j\cdot k\cdot z$ so $z|z$ and $x=j\cdot k\cdot x$ so $x|x$ so $x=z$. And 
            $y-w+w-y\geq 0$ and $w-y+y-w\geq 0$ so $y=w$.
            \item
            Thus $(x,y)=(z,w)$ as required.
        \end{enumerate}
        There exists an $x$ and $y$ such that $(x,y)\cancel{R}(z,w)$ and $(z,w)\cancel{R}(x,y)$
        \begin{enumerate}[label=\quad\quad, leftmargin=*]
            \item 
            Consider $(x,y)=(4,2)$ and $(z,w)=(3,5)$ then it can be seen that $4\cdot j\neq 3$ for some
            $j\in\mathbb{Z}^+$ and $2-5\ngeq 0$. And $3\cdot k\neq 4$ for some $k\in\mathbb{Z}^+$.
        \end{enumerate}
    \end{proof}

    %###################################################################################

    \section*{Problem 7}

    Give an example of a partially ordered set which contains exactly one minimal element 
    $x$ such that $x$ is not the smallest element.

    Let set $G=\{(x,y)\;|\; (x=2 \land y=1) \lor (x=1 \land y\in\mathbb{N})\}$ and let relation 
    $R=\{((x_1,y_1),(x_2,y_2))\;|\; (x_1>x_2\land y_1=y_2=1)\lor (x_1=x_2=1\land y_1>y_2)\rightarrow (x_1,y_1) < (x_2,y_2)\}$.
\end{document}