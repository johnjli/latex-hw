\documentclass{article}
\usepackage[utf8]{inputenc}
\usepackage{amsmath}
\usepackage{amssymb}
\usepackage{amsthm}
\usepackage[makeroom]{cancel}

\setlength{\parskip}{1em}
\setlength\parindent{0px}
\title{Homework 4}
\date{\today}
\author{John J Li}

\begin{document}
    \pagenumbering{gobble}
    \maketitle
    \newpage
    \pagenumbering{arabic}

    %###################################################################################

    \textbf{Problem 1:}

    Prove that for every integer $n$, $n^3$ is odd if and only if $n$ is odd.

    \begin{proof}
        1) $n^3$ is odd $\rightarrow$ $n$ is odd
        
        Indirectly. Suppose $n$ is even. Then $n=2k$ for some $k\in\mathbb{Z}$.
        Therefore $n^3=(2k)^3=8k^3=2(4k^3)$ and $4k^3 \in\mathbb{Z}$. Thus $n^3$ is even.

        2) $n$ is odd $\rightarrow$ $n^3$ is odd

        Suppose $n$ is odd. Then $n=2k+1$ for some $k\in\mathbb{Z}$. Therefore 
        $n^3=(2k+1)^2=8k^3+6k^2+6k+1=2(4k^3+6k^2+3k)+1$ and $4k^3+6k^2+3k \in\mathbb{Z}$.
        Thus $n^3$ is odd.
    \end{proof}

    %###################################################################################

    \textbf{Problem 2:}

    Prove that for every two integers $m,n$, $m+n$ is even if and only if $3m^2+n^3$
    is even.

    \begin{proof}
        \textbf{1)} $m+n$ is even $\rightarrow$ $3m^2+n^3$ is even.

        (Case 1) Suppose $m$ is even and $n$ is even.

        Then it could be said that $m = 2k$ and $n=2j$ for some $k,j\in\mathbb{Z}$. 
        Then $3(2k)^2 + (2j)^3$ and then $12k+8j$ which can be written: $2(6k+4j)$ which
        is even.

        (Case 2) Suppose $m$ is odd and $n$ is odd.

        Then it could be said that $m=2k+1$ and $n=2j+1$ for some $k,j\in\mathbb{Z}$.
        Then $3(2k+1)^2+(2j+1)^3$ and then $12k^2+12k+8j^3+12j^2+6j+4$ which can be written:
        $2(6k^2+6k+4j^3+6j^2+3j+2)$ which is even.

        \textbf{2)} $3m^2+n^3$ is even $\rightarrow$ $m+n$ is even.

        Indirectly. Suppose $m+n$ is odd. 

        (Case 1) Suppose $m$ is odd and $n$ is even

        Then it could be said that $m=2k+1$ and $n=2j$ for some $k,j\in\mathbb{Z}$.
        Then $3(2k+1)^2+(2j)^3$ and then $12k^2+12k+8j^3+3$ which can be written:
        $2(6k^2+6k+4j^3+1)+1$ which is odd.

        (Case 2) Suppose $m$ is even and $n$ is odd

        Then it could be said that $m=2k$ and $n=2j+1$ for some $k,j\in\mathbb{Z}$.
        Then $3(2k)^2+(2j+1)^3$ and then $12k^2+12k+8j^3+12j^2+6j+1$ which can be written:
        $2(6k^2+6k+4j^3+6j^2+3j)+1$ which is odd.
    \end{proof}

    %###################################################################################

    \textbf{Problem 3:}

    Prove that for every $x\in\mathbb{R}$, $|x-4|>4$, then $x^2 > 8x$.

    \begin{proof}
        Indirectly. Suppose $x^2\leq 8x$. This implies $0\leq x^2\leq 8x$ and by dividing 
        by $x$, the equation becomes: $0\leq x \leq 8$ so that adding $-4$ to all sides gives
        $-4\leq x-4 \leq 4$ which is the same as $|x-4|\leq 4$.
    \end{proof}

    %###################################################################################

    \textbf{Problem 4:}

    (a) Prove that $\mathcal{P}(A) \cup \mathcal{P}(B) \subseteq \mathcal{P}(A\cup B)$

    \begin{proof}
        Suppose $x\in \mathcal{P}(A) \cup \mathcal{P}(B)$ then $x\subseteq A$ or 
        $x\subseteq B$. Which can be written as $x\subseteq A\cup B$ which implies
        $x\in \mathcal{P}(A\cup B)$. Thus, since $x\in \mathcal{P}(A) \cup \mathcal{P}(B)$
        and $x\in \mathcal{P}(A\cup B)$, 
        $\mathcal{P}(A) \cup \mathcal{P}(B) \subseteq \mathcal{P}(A\cup B)$.
    \end{proof}

    (b) Prove that $P(A \cup B) = P(A) \cup P(B)$ if and only if either 
    $A \subseteq B$ or $B \subseteq A$.

    \begin{proof}
        Indirectly. Suppose $A\nsubseteq B$ and $B\nsubseteq A$. There there exists
        $a\in A\setminus B$ and $b\in B\setminus A$. 

        Let $x=\{a,b\}$. 
        \begin{itemize}
            \item Then $x\subseteq A\cup B$ and so $x\in\mathcal{P}(A\cup B)$.
            \item Then $x\notin \mathcal{P}(A)$ because $b\in B\setminus A$.
            \item Then $x\notin \mathcal{P}(B)$ because $a\in A\setminus B$.
            \item Thus $x\notin \mathcal{P}(A)\cup\mathcal{P}(B)$.
        \end{itemize}
        Thus $\mathcal{P}(A\cup B) \neq \mathcal{P}(A)\cup\mathcal{P}(B)$.
    \end{proof}

    %###################################################################################

    \textbf{Problem 5:}

    Let $A,\; B$ be sets. Prove that $A \setminus (A \setminus B) = A \cap B$.

    \begin{proof}

        Suppose $y,x\in A$ and $z,x\in B$ but $y\notin B$ and $z\notin A$. Then
        $x\in (A \setminus (A \setminus B))$ but $y,z\notin (A \setminus (A \setminus B))$.
        So it can be said that for every element in $A$ that is not in $B$ will not be in
        $A \setminus (A \setminus B)$ and vice versa for $B$, but for every element in both
        $A$ and $B$, it will also be in $A \setminus (A \setminus B)$ which fullfils the
        definition of $A\cap B$. Thus $A \setminus (A \setminus B) = A \cap B$.
    \end{proof}

    %###################################################################################

    \textbf{Problem 6:}

    Prove that for any integers $a,\; b,\; c$ if $a|b$ or $a|c$ then $a|(c \cdot b)$.

    \begin{proof}
        1) Suppose $a|b \rightarrow a|(c\cdot b)$.

        Then $a(i) = b$ for some $i \in\mathbb{Z}$. Then suppose each side is multiplied by 
        some integer $c$ such that $a(i)c = (b\cdot c)$. Since $i,c \in\mathbb{Z}$,
        it can be written: $a|(b\cdot c)$. So if $a|b$ then 
        $a|(c \cdot b)$.

        2) Suppose $a|c \rightarrow a|(c\cdot b)$.

        Then $a(j) = c$ for some $j \in\mathbb{Z}$. Then suppose each side is multiplied by 
        some integer $b$ such that $a(j)b = (b\cdot c)$. Since $j,c \in\mathbb{Z}$,
        it can be written: $a|(b\cdot c)$. So if $a|c$ then 
        $a|(c \cdot b)$.

        3) Suppose $a|b$ and $a|c$ $\rightarrow a|(c\cdot b)$
        
        Then $a(i) = b$ and $a(j)=c$ for some $i,j \in\mathbb{Z}$.
        Then $b\cdot c=a(i)\cdot a(j)$ and so $b\cdot c=a^2ij$. Since $i,j$ are integers,
        $a^2|(b\cdot c)$ which implies $a|(b\cdot c)$. So if $a|b$ or $a|c$ then 
        $a|(c \cdot b)$.
    \end{proof}

    %###################################################################################

    \textbf{Problem 7:}

    Prove that for every integer $n$, $15|n$ if and only if $3|n$ and $5|n$.

    \begin{proof}
        1) $15|n \rightarrow$ $3|n$ and $5|n$

        Suppose $15|n$ then $15(k) = n$ for some $k\in\mathbb{Z}$. And since $15 = 3\cdot 5$,
        then $3(5)k=n$, since $5k$ is an integer it can be subsitute for $j\in\mathbb{Z}$.
        Then $3j=n$ which implies $3|n$. Subsequently, $3k$ can be subsituted for
        $i\in\mathbb{Z}$ since $3k$ is an integer, thus giving $5i=n$ which implies $5|n$.
        So if $15|n$ then $3|n$ and $5|n$.

        2) $3|n$ and $5|n \rightarrow 15|n$

        Suppose $3|n$ and $5|n$ then $3j=n$ and $5i=n$ for some $j,i\in\mathbb{Z}$. 
        Multiply $3j=n$ and $5i=n$ together to get $15ij=n^2$. 
        Since $j,i\in\mathbb{Z}$ they can be subsituted for $k\in\mathbb{Z}$ to get
        $15k=n^2$ which is $15|n^2$ which implies $15|n$. And so if $3|n$ and $5|n$
        then $15|n$.
    \end{proof}

    %###################################################################################

    \textbf{Problem 8:}

    Show that $\sqrt{3}$ is irrational.

    \begin{proof}
        By contradiction. 
        
        Suppose $\sqrt{3}$ is rational. Then $\sqrt{3} = \frac{p}{q}$
        where $p$ and $q$ are integers, share no common factors, and $q\neq 0$. Then
        $3=\frac{p^2}{q^2}$ and so $3q^2=p^2$. So $3|p^2$ which implies $3|p$ and $3k=p$
        where $k \in\mathbb{Z}$ Thus $3q^2=(3k)^2$ then $q^2=3k$ so $3|q^2$ which implies
        $3|q$. And so $q=3j$ for some $j\in\mathbb{Z}$. 
        
        This contradicts our statement that $p$ and $q$ share no common factors. So, there 
        are no integers $p$ and $q$ such that $\sqrt{3}$ is rational by contradiction. Thus,
        $\sqrt{3}$ must be irrational.
    \end{proof}

    %###################################################################################

    \textbf{Problem 9:}

    Let $U$ be a nonempty set. Show that there is a unique set $A \in \mathcal{P}(U)$ such
    that for every set $B \in\mathcal{P}(U)$, $A \cap B = B$.

    \begin{proof}
        Existence: Since both $A,B\in\mathcal{P}(U)$, it can be stated that 
        $\forall B\; (U\cap B=B)$, so $U$ has the required property. 

        Uniqueness: Suppose there are arbitrary sets $C,D\in\mathcal{P}(U)$ and $\forall B\; (C\cap B=B)$
        and  $\forall B\; (D\cap B=B)$. Applying the first assumption to $D$ it is seen that
        $C\cap D = D$, and applying the second to $C$, we get $D\cap C = C$. Clearly,
        $C\cap D = D\cap C$, so $C=D$.
    \end{proof}

    %###################################################################################

    \textbf{Problem 10:}

    Prove that if 
    $\lim_{x\rightarrow a} f(x) = C$ and $\lim_{x\rightarrow a} g(x) = D$, then 
    $\lim_{x\rightarrow a}(f(x)+g(x)) = C + D$.

    \begin{proof}
        Suppose $\lim_{x\rightarrow a} f(x) = C$ and $\lim_{x\rightarrow a} g(x) = D$.

        Let $\epsilon_1 >0$ and $\delta_1 >0$. Let $x\in\mathbb{R}$ and if $0<|x-a|<\delta_1$
        then $|f(x)-C|<\epsilon_1$. Let $\epsilon_2 >0$ and $\delta_2 >0$. Let $x\in\mathbb{R}$ 
        and if $0<|x-a|<\delta_2$ then $|g(x)-D|<\epsilon_2$. Consider $\epsilon_1$ and 
        $\epsilon_2$ to be $\frac{\epsilon}{2}$ and $\delta$ to be smaller than 
        $\delta_1$ and $\delta_2$ so that if $0<|x-a|<\delta$
        then $|f(x)-C|<\frac{\epsilon}{2}$ and if $0<|x-a|<\delta$ then 
        $|g(x)-D|<\frac{\epsilon}{2}$.

        After adding both inequalities together, if $0<|x-a|<\delta$ then 
        $|f(x)-C|+|g(x)-D|<2\cdot\frac{\epsilon}{2}=\epsilon$. 
        
        Because $|f(x)-C|$ and $|g(x)-D|$ $\geq |f(x)-C + g(x)-D|$, 
        it can be written: if $0<|x-a|<\delta$ then 
        $|f(x)-C + g(x)-D|\leq |f(x)-C|+|g(x)-D|<\epsilon$, and so
        if $0<|x-a|<\delta$ then $|f(x)-C + g(x)-D|<\epsilon$.

        Then if $0<|x-a|<\delta$ then $|(f(x) + g(x))-(D+C)|<\epsilon$ which is the 
        definition of $\lim_{x\rightarrow a} (f(x)+g(x)) = C+D$. So if 
        $\lim_{x\rightarrow a} f(x) = C$ and $\lim_{x\rightarrow a} g(x) = D$, then 
        $\lim_{x\rightarrow a}(f(x)+g(x)) = C+D$.
    \end{proof}

    

\end{document}