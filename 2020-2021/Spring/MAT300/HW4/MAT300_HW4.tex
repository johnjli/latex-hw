\documentclass{article}
\usepackage[utf8]{inputenc}
\usepackage{amsmath}
\usepackage{amssymb}
\usepackage{amsthm}
\usepackage[makeroom]{cancel}

\setlength{\parskip}{1em}
\setlength\parindent{0px}
\title{Homework 3}
\date{\today}
\author{John J Li}

\begin{document}
    \pagenumbering{gobble}
    \maketitle
    \newpage
    \pagenumbering{arabic}

    %###################################################################################

    \textbf{Problem 1:}

    \begin{proof}
        1) $n^3$ is odd $\rightarrow$ $n$ is odd
        
        Indirectly. Suppose $n$ is even. Then $n=2k$ for some $k\in\mathbb{Z}$.
        Therefore $n^3=(2k)^3=8k^3=2(4k^3)$ and $4k^3 \in\mathbb{Z}$. Thus $n^3$ is even.

        2) $n$ is odd $\rightarrow$ $n^3$ is odd

        Suppose $n$ is odd. Then $n=2k+1$ for some $k\in\mathbb{Z}$. Therefore 
        $n^3=(2k+1)^2=8k^3+6k^2+6k+1=2(4k^3+6k^2+3k)+1$ and $4k^3+6k^2+3k \in\mathbb{Z}$.
        Thus $n^3$ is odd.
    \end{proof}

    %###################################################################################

    \textbf{Problem 2:}

    \begin{proof}
        
    \end{proof}

    %###################################################################################

    \textbf{Problem 3:}

    \begin{proof}
        
    \end{proof}

    %###################################################################################

    \textbf{Problem 4:}

    \begin{proof}
        
    \end{proof}

    %###################################################################################

    \textbf{Problem 5:}

    \begin{proof}
        
    \end{proof}

    %###################################################################################

    \textbf{Problem 6:}

    \begin{proof}
        
    \end{proof}

    %###################################################################################

    \textbf{Problem 7:}

    \begin{proof}
        
    \end{proof}

    %###################################################################################

    \textbf{Problem 8:}

    \begin{proof}
        By contradiction. 
        
        Suppose $\sqrt{3}$ is rational. Then $\sqrt{3} = \frac{p}{q}$
        where $p$ and $q$ are integers, share no common factors, and $q\neq 0$. Then
        $3=\frac{p^2}{q^2}$ then $3q^2=p^2$. So $3|p^2$ which implies $3|p$ and $3k=p$
        where $k \in\mathbb{Z}$ Thus $3q^2=(3k)^2$ then $q^2=3k$ so $3|q^2$ which implies
        $3|q$. And so $q=3j$ for some $j\in\mathbb{Z}$.
        This contradicts our statement that $p$ and $q$ share no common factors. So, there 
        are no integers $p$ and $q$ such that $\sqrt{3}$ is rational by contradiction. Thus,
        $\sqrt{3}$ must be irrational.

    \end{proof}

    %###################################################################################

    \textbf{Problem 9:}

    \begin{proof}
        
    \end{proof}

    %###################################################################################

    \textbf{Problem 10:}

    \begin{proof}
        
    \end{proof}

    

\end{document}