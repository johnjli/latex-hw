\documentclass{article}

\usepackage[a4paper, total={6in, 8in}]{geometry}
\usepackage[utf8]{inputenc}
\usepackage{amsmath}
\usepackage{amssymb}
\usepackage{amsthm}
\usepackage{fancyhdr}
\usepackage{enumitem}

\pagestyle{fancy}
\fancyhf{}
\lhead{John J Li}
\rhead{MAT300 Spring 2021 Homework 5}
\rfoot{\thepage}
\lfoot{March 2021}
\renewcommand{\headrulewidth}{0.4pt}
\renewcommand{\footrulewidth}{0.4pt}

\setlength{\parskip}{1em}
\setlength\parindent{0px}
\title{MAT300 Spring 2021 Homework 5}
\date{\today}
\author{John J Li}

\begin{document}
    \maketitle
    \thispagestyle{empty}
    \noindent\rule{\textwidth}{0.8pt}

    %###################################################################################

    \section*{Problem 1}

    Show that for every $n\in\mathbb{N}, \sum^n_{i=1} i^2 = \frac{n(n+1)(2n+1)}{6}$.

    \textbf{Solution:}

    \begin{proof}
        Using mathematical induction.

        Base case: Setting $n=1$, it is seen that $1^2=\frac{1((1)+1)(2(1)+1)}{6}=1$ as required.

        Induction step: Let n be an arbitrary natural number and suppose that $\sum^n_{i=1} i^2 = \frac{n(n+1)(2n+1)}{6}$.
        Then, 
        \begin{equation*}
            \begin{split}
                1^2+2^2+...+(n+1)^2 & =(1^2+2^2+...+n^2) + (n+1)^2 \\
                & = \frac{n(n+1)(2n+1)}{6} + (n+1)^2 \\
                & = \frac{n(n+1)(2n+1)+6(n+1)^2}{6} \\
                & = \frac{2n^3+9n^2+13n+6}{6} \\
                & = \frac{(n+1)(n+2)(2n+3)}{6}
            \end{split}
        \end{equation*}
    \end{proof}

    %###################################################################################

    \section*{Problem 2}

    Show that for every $n\in\mathbb{N}, \sum^n_{i=0} (-\frac{1}{2})^i=\frac{2^{n+1}+(-1)^n}{3\cdot 2^n}$.

    \textbf{Solution:}

    \begin{proof}
        Using mathematical induction.

        Base case: Setting $n=0$, it is seen that $(-\frac{1}{2})^0=\frac{2^1+(-1)^0}{3\cdot 2^0} = 1$
        as required.

        Induction step: Let $n$ be an arbitrary natural number and suppose $\sum^n_{i=0} (-\frac{1}{2})^i=\frac{2^{n+1}+(-1)^n}{3\cdot 2^n}$.
        Then, 
        \begin{equation*}
            \begin{split}
                \left(-\frac{1}{2}\right)^0+\left(-\frac{1}{2}\right)^1+...+
                \left(-\frac{1}{2}\right)^{n+1} & = \left(\left(-\frac{1}{2}\right)^0+
                \left(-\frac{1}{2}\right)^1+... +\left(-\frac{1}{2}\right)^{n}\right) + 
                \left(-\frac{1}{2}\right)^{n+1} \\
                & = \frac{2^{n+1}+(-1)^n}{3\cdot 2^n} +\left(-\frac{1}{2}\right)^{n+1} \\
                & = \frac{(2)^{n+1}\left(2^{n+1}+(-1)^n\right)}{3(2)^n(2)^{n+1}} + 
                \frac{3(2)^n(-1)^{n+1}}{3(2)^n(2)^{n+1}} \\
                & = \frac{(2)^n2^{n+2}+2^{n+1}(-1)^n-3(2)^n(-1)^n}{3(2)^n2^{n+1}} \\
                & = \frac{(2)^n2^{n+2}+(2(2^n)-3(2)^n)(-1)^n}{3(2)^n2^{n+1}} \\
                & = \frac{(2)^n2^{n+2}+2^n(2-3)(-1)^n}{3(2)^n2^{n+1}} \\
                & = \frac{2^{n+2}+(-1)^{n+1}}{3(2)^{n+1}}
            \end{split}
        \end{equation*}
    \end{proof}

    %###################################################################################

    \section*{Problem 3}

    Show that for every $n\in\mathbb{N}, 9|(4^n+6n-1)$.

    \begin{proof}
        Using mathematical induction.

        Base case: If $n=0$, then $4^n+6n-1 = 0 = 9\cdot 0$, so $9|(4^n+6n-1)$.

        Induction step: Let $n$ be an arbitrary natural number and suppose $9|(4^n+6n-1)$.
        Then $9k=4^n+6n-1$ for some integer $k$. Thus,
        \begin{equation*}
            \begin{split}
                4^{n+1}+6(n+1)-1 & = 4(4)^n+6n+5 \\
                & = 4(4)^n+24n-4-18n+9 \\
                & = 4(4^n+6n-1)-18n+9 \\
                & = 4(9k)-18n+9 \\
                & = 9(4k-2n+1)
            \end{split}
        \end{equation*}

        Therefore $9|(4^{n+1}+6(n+1)-1)$, as required.
    \end{proof}

    %###################################################################################

    \section*{Problem 4}

    Show that for all non-negative integers $n, 3|(2^{2n}-1)$.

    \begin{proof}
        Using mathematical induction.

        Base case: If $n = 0$, then $2^0-1 = 0 = 3\cdot 0$, so $3|(2^{2n}-1)$.

        Induction step: Let $n$ be an non-negative integer and suppose $3|(2^{2n}-1)$.
        Then $3k=2^{2n}-1$ for some integer $k$. Thus,
        \begin{equation*}
            \begin{split}
                2^{2n+2}-1 & = 4(2)^{2n}-1 \\
                & = 4(2)^{2n}-4+3 \\ 
                & = 4(2^{2n}-1)+3 \\
                & = 4(3k)+3 \\
                & = 3(4k+1)
            \end{split}
        \end{equation*}
        
        Therefore $3|(2^{2n+2}-1)$, as required.
    \end{proof}

    %###################################################################################

    \section*{Problem 5}

    For all non-negative integers $n, 101|(10^{2n}-(-1)^n)$.

    \begin{proof}
        Using mathematical induction.

        Base case: If $n = 0$, then $10^0-(-1)^0 = 0 = 101\cdot 0$, so $101|(10^{2n}-(-1)^n)$.

        Induction step: Let $n$ be an non-negative integer and suppose $101|(10^{2n}-(-1)^n)$.
        Then $101k=10^{2n}-(-1)^n$ for some integer $k$. Thus,
        \begin{equation*}
            \begin{split}
                10^{2n+2}-(-1)^{n+1} & = 100(10)^{2n}+(-1)^n \\
                & = 100(10)^{2n}-100(-1)^n+101(-1)^n \\
                & = 100(10^{2n}-(-1)^n)+101(-1)^n \\
                & = 100(101k)+101(-1)^n \\
                & = 101(100k+(-1)^n)
            \end{split}
        \end{equation*}
        
        Therefore $101|(10^{2n+2}-(-1)^{n+1})$, as required.
    \end{proof}

    %###################################################################################

    \section*{Problem 6}

    For all non-negative integers $n, n^2\leq 3^n$.
    
    \begin{proof}
        Using mathematical induction.

        Base case: When $n=0$, it is seen that $n^2 = 0 \leq 1 = 3^n$.

        Induction step: Let $n$ be an arbitrary non-negative integer and suppose $n^2\leq 3^n$
        Then,
        \begin{align*}
            (n+1)^2 & = n^2+2n+1 \\
            & \leq n^2 + n^2 + n^2+1 \\
            & = 3(n)^2+1 \\ 
            & \leq 3(3)^n && \text{(inductive hypothesis)} \\
            & = 3^{n+1}.
        \end{align*}
    \end{proof}

    %###################################################################################

    \section*{Problem 7}

    Show that for $n\in\mathbb{Z}^+$ if $n\geq 5$ then $(n+1)^2\leq n!$.

    \begin{proof}
        Using mathematical induction.

        Base case: When $n=5$, it is seen that $(5+1)^2=36\leq 120=5!$.

        Induction step: Let $n\geq 5$ be arbitrary, and suppose $(n+1)^2\leq n!$.
        Then,
        \begin{align*}
            (n+2)^2 & = (n+1)^2+2n+3 \\
            & \leq n!+2n+3 && \text{(inductive hypothesis)} \\
            & \leq (n+1)n! && (\text{since }n\geq 5) \\
            & = (n+1)!
        \end{align*}
    \end{proof}
    
    %###################################################################################

    \section*{Problem 8}

    Show that for every $n\in\mathbb{N}$ and $h>-1, 1+nh\leq(1+h)^n$.
    
    \begin{proof}
        Using mathematical induction.

        Base case: When $n=0$ and $h=0$, it is seen that $1+(0)(0)=1\leq 1 =(1+(0))^0$.

        Induction step: Let $n$ be an arbitrary natural number, and
        suppose $1+nh\leq (1+h)^n$. 
        Then proving $1+(n+1)h\leq (1+h)^{n+1}$,
        \begin{align*}
            1+(n+1)h & \leq 1+(n+1)h+nh^2 && (\text{since } h>-1)\\
            & = (1+nh)(1+h) \\
            & \leq (1+h)^n(1+h) && (\text{inductive hypothesis})\\
            & = (1+h)^{n+1}
        \end{align*}
    \end{proof}

    %###################################################################################

    \section*{Problem 9}

    Use mathematical induction to show that for $n\in\mathbb{Z}^+$, the number of 3-element
    subsets of the set $\{1,...,n\}$ is $n(n-1)(n-2)/6$.

    \begin{proof}
        Since $n=1,2$ yields no possible 3-element subsets, $n=3$ will be the base case.
        Let $n=3$, then the number of 3-element subsets is $1=1=3(3-1)(3-2)/6$.

        Let $\{1,...,n\}$ be an arbitrary set, and suppose $\frac{n(n-1)(n-2)}{6}$.
        Then proving the number of 3-element subsets of the set with an additional element requires
        two cases, where the first case is when $n+1$ is an element of the 3-element subset and
        the second case being when $n+1$ is not an element of the 3-element subset.

        Case 1: Suppose $n+1$ is an element of the 3-element subset. Thus there exists two 
        other elements in $\{1,...,n\}$. 

        Case 2: Suppose $n+1$ is not an element of the 3-element subset. Then there must
        exists 3 elements in $\{1,...,n\}$ which makes up the 3-element subset. 
    \end{proof}

\end{document}