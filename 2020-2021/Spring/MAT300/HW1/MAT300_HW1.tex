\documentclass{article}
\usepackage[utf8]{inputenc}
\usepackage{amsmath}
\usepackage{amssymb}
\usepackage[makeroom]{cancel}

\setlength{\parskip}{1em}
\setlength\parindent{0px}
\title{Homework 1}
\date{2021 January 26}
\author{John J Li}

\begin{document}
    \pagenumbering{gobble}
    \maketitle
    \newpage
    \pagenumbering{arabic}

    \textbf{Problem 1:}

    \quad \textbf{(a)} $((p\land q)\lor p) \land q$

    \begin{gather*}
        ((p\land q)\lor p) \land q \\
        ((p \lor p) \land (q \lor p)) \lor p \\
        (\cancel{(p \lor p)} \land (q \lor p)) \lor p \\
        q\land q \lor p \land q \\
        \cancel{q\land q} \lor p \land q \\
        p \land q
    \end{gather*}

    \fbox{%
        \parbox{\textwidth}{%
            Not a tautology because $p\land q$ can be false.
        }%
    }

    \quad\textbf{(b)} $p\rightarrow (p\lor q)$

    \begin{gather*}
        p\rightarrow (p\lor q) \\
        \neg p \lor (p \lor q) \\
        (\neg p \lor p) \lor (\neg p \lor q)
    \end{gather*}

    \fbox{%
        \parbox{\textwidth}{%
            This is a tautology because $\neg p \lor p$ is always true and so
            $(\neg p \lor p) \lor (\neg p \lor q)$ is always true as well.
        }%
    }

    \quad\textbf{(c)} $(\neg p \lor q) \rightarrow (p \rightarrow q)$

    \begin{gather*}
        \text{Let} \; A \equiv (p \rightarrow q) \\
        (\neg p \lor q) \rightarrow (p \rightarrow q) \\
        (p \rightarrow q) \rightarrow (p \rightarrow q) \\
        A \rightarrow A
    \end{gather*}

    \fbox{%
        \parbox{\textwidth}{%
            This is a tautology because the statement is conditionally comparing the same things.
        }%
    }

    \textbf{Problem 2:}

    \quad \textbf{(a)} $x \in \mathbb{R}$ and if $x<5$, then $x=5$

    \quad\quad$\boxed{\{x \in \mathbb{R}\;|\; x<5 \rightarrow x=5\} = \{x \geq 5\}}$

    \quad \textbf{(b)} $z \in \mathbb{R}$ and $2 \in \{x \in \mathbb{R} \; | \; x \cdot z < 2\}$

    \quad\quad$\boxed{\{ z \in \mathbb{R} \; | \; 2 \in \{x \in \mathbb{R} \; | \; x \cdot z < 2\}\} =\{z < 2 \}}$

    \quad \textbf{(c)} $n \in \mathbb{Z}$ and $n \in \{2^i \; | \; i \in \mathbb{Z}^+\}$

    \quad\quad$\boxed{\{0,1,2,3...\}}$

    \textbf{Problem 3:}

    \quad \textbf{(a)} $x \notin A \setminus (B\cup C)$

    \begin{gather*}
        x \notin A \setminus (B\cup C) \\
        x \notin A \land \neg(x \notin B \lor x\notin C) \\
        x \notin A \land x \in B \land x \in C
    \end{gather*}

    \quad\quad $\boxed{x \notin A \land x \in B \land x \in C}$

    \quad \textbf{(a)} $x \in A \setminus (B\setminus C)$

    \begin{gather*}
        x \in A \setminus (B\setminus C) \\
        x \in A \land \neg(x \in B \land \neg(x\in C)) \\
        x \in A \land (x \notin B \lor x \in C)
    \end{gather*}

    \quad\quad $\boxed{x \in A \land (x \notin B \lor x \in C)}$

    \textbf{Problem 4:}

    \quad \textbf{(a)} $\emptyset \subseteq \{\emptyset\}$

    \fbox{%
        \parbox{\textwidth}{%
            This statement is true.
        }%
    }

    \quad \textbf{(b)} $(\{1,\emptyset\}\cap\{2,\emptyset\}) \subseteq \{1,2\}$

    \fbox{%
        \parbox{\textwidth}{%
            This statement is false because $(\{1,\emptyset\}\cap\{2,\emptyset\}) \equiv \{\emptyset\}$
            which is not a subset of $\{1,2\}$.
        }%
    }

    \textbf{Problem 5:}

    \quad \textbf{(a)} $A\setminus (B\cup C) = (A\setminus B) \cap (A \setminus C)$

    \begin{gather*}
        A\setminus (B\cup C) = (A\setminus B) \cap (A \setminus C) \\
        A \land \neg (B \lor C) = (A \land \neg B) \land (A \land \neg C) \\
        \boxed{A \land \neg B \land \neg B = A \land \neg B \land \neg B}
    \end{gather*}

    \quad \textbf{(b)} $(A\setminus B) \cup (B\setminus A) = (A \cup B) \setminus (A \cap B)$

    \begin{gather*}
        (A\setminus B) \cup (B\setminus A) = (A \cup B) \setminus (A \cap B) \\
        (A \land \neg B) \lor (B \land \neg A) = (A \lor B) \land \neg(A \land B) \\
        (A\lor B) \land (A \lor \neg A) \land (\neg B \lor B) \land (\neg B \lor \neg A) = (A \lor B) \land (\neg A \lor \neg B) \\
        (A\lor B) \land \cancel{(A \lor \neg A)} \land \cancel{(\neg B \lor B)} \land (\neg B \lor \neg A) = (A \lor B) \land (\neg A \lor \neg B) \\
        \boxed{(A \lor B) \land (\neg A \lor \neg B) = (A \lor B) \land (\neg A \lor \neg B)}
    \end{gather*}

    \textbf{Problem 6:}

    \begin{gather*}
        (p\rightarrow q) \land (\neg p \rightarrow q) \equiv p \leftrightarrow q \\
        (\neg p \lor q) \land (p \lor \neg q) \equiv (p \rightarrow q) \land (q \rightarrow p) \\
        \boxed{(\neg p \lor q) \land (\neg q \lor p) \equiv (\neg p \lor q) \land (\neg q \lor p)}
    \end{gather*}

    \textbf{Problem 7:}

    \quad \textbf{(a)} If $x+y < 2$ and $y+z = 2$, then $x^2 + y^2 + z^2 = 2$ or $x^2 \geq y^2$.

    \quad\quad Converse:

    \fbox{%
        \parbox{\textwidth}{%
            If $x^2 + y^2 + z^2 = 2$ or $x^2 \geq y^2$, then $x+y < 2$ and $y+z = 2$.
        }%
    }

    \quad\quad Contrapositive:

    \fbox{%
        \parbox{\textwidth}{%
        If $x+y \leq 2$ or $y+z \neq 2$, then $x^2 + y^2 + z^2 \neq 2$ and $x^2 < y^2$
        }%
    }

    \quad \textbf{(b)} If two integers $x,y$ are both even, then at least one of $x^2 + y,\; y^2 +x$
    is even.

    \quad\quad Converse:

    \fbox{%
        \parbox{\textwidth}{%
            If at least one of $x^2 + y,\; y^2 +x$ is true, then the two integers $x,y$ are both even.
        }%
    }

    \quad\quad Contrapositive:

    \fbox{%
        \parbox{\textwidth}{%
            If two integers $x,y$ are both odd, then both $x^2 + y,\; y^2 +x$
            are odd.
        }%
    }

    \textbf{Problem 8:}

    \quad \textbf{(a)} $P(x) : x < 2 \rightarrow x > 2$

    \quad\quad $\boxed{3}$
    
    \quad \textbf{(b)} $P(x) : x \in \mathbb{Z}^+ \rightarrow x = 0$

    \quad\quad $\boxed{-1}$

    \quad \textbf{(c)} $P(x) : (x^2 +1 = x) \rightleftarrows x < 2$

    \quad\quad $\boxed{-1}$

    \textbf{Problem 9:}

    \quad \textbf{(a)} X says "Both of us are knaves".

    \fbox{%
        \parbox{\textwidth}{%
            X is a knave while Y is a knight. This is because this statement is always false -- knaves cannot tell the truth so
            we know X and Y are not both knaves. Therefore, X and Y could both be knights but since
            we know X is lying, we can conclude that Y is a knight.
        }%
    }
    
    \quad \textbf{(b)} Y says "Either I am a knave or X is a knight"

    \fbox{%
        \parbox{\textwidth}{%
            X and Y can be both knights or Y is a knave and X is a knight.
        }%
    }

    \quad \textbf{(c)} X says "I am a knave but Y isn't"

    \fbox{%
        \parbox{\textwidth}{%
            This statement is always false; we can automatically conclude that y is also a knave.
            So both X and Y are knaves -- since knights can't say they are knaves.
        }%
    }

    \textbf{Problem 10:}

    \fbox{%
        \parbox{\textwidth}{%
            The number of words we use to define the smallest positive integer which cannot
            be described in fewer than sixteen words is 16. 16 is a subset of A since we
            can described 16 in fewer than sixteen words.
        }%
    }

\end{document}