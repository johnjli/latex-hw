\documentclass{article}
\usepackage[utf8]{inputenc}
\usepackage{amsmath}
\usepackage{amssymb}
\usepackage[makeroom]{cancel}

\setlength{\parskip}{1em}
\setlength\parindent{0px}
\title{Homework 2}
\date{\today}
\author{John J Li}

\begin{document}
    \pagenumbering{gobble}
    \maketitle
    \newpage
    \pagenumbering{arabic}

    %###################################################################################

    \textbf{Problem 1:}

    \quad \textbf{(a)} $(\forall_x \, x < 2) \rightarrow (\exists_x x > 2)$

    \fbox{%
        \parbox{0.95\textwidth}{%
            The truth value for this statement is false since x can be a value less than 2.
        }%
    }

    \quad \textbf{(b)} $(\forall_x \, x < 0) \longleftrightarrow (\exists_x x^2 + 2 < 0)$

    \fbox{%
        \parbox{0.95\textwidth}{%
            The truth value for this statement is false because there doesn't exist an x where
            $\exists_x x^2 + 2 < 0$ would be true. Thus making $(\forall_x \, x < 0) \rightarrow (\exists_x x^2 + 2 < 0)
            \land (\exists_x x^2 + 2 < 0) \rightarrow (\forall_x \, x < 0)$ false.
        }%
    }

    \quad \textbf{(c)} $\forall_x (x < 0 \rightarrow x^2 = 1)$

    \fbox{%
        \parbox{0.95\textwidth}{%
            The truth value for this statement is false since x can be any value less than 0
            and where $x^2 \neq 1$ such as $x = -2$
        }%
    }

    \quad \textbf{(d)} $\exists_x (x < 0 \rightarrow x^2 = 1)$

    \fbox{%
        \parbox{0.95\textwidth}{%
            The truth value for this statement is true since $x = -1$ would make this statement
            true.
        }%
    }

    %###################################################################################

    \textbf{Problem 2:}

    \quad \textbf{(a)} $\forall_x\exists_y \, x + y = 5$

    \fbox{%
        \parbox{0.95\textwidth}{%
            The truth value for this statement is true. Consider $y_x = 5-x$, thus $y\in \mathbb{Z}$
            and $x+y_x = x + 5 - x = 5$.
        }%
    }

    \quad \textbf{(b)} $\exists_x\forall_y \, x + y = 5$

    \fbox{%
        \parbox{0.95\textwidth}{%
            The truth value for this statement is false because for a certain x, not all y
            would make $x+y=5$ true. Consider $y_x = 6 - x$, then $y_x \in \mathbb{Z}$ 
            and $x + y_x = 6 \neq 5$.
        }%
    }

    \quad \textbf{(c)} $\forall_x\exists_y \, (x + y = 5 \rightarrow x^2 + y^3 > 7)$

    \fbox{%
        \parbox{0.95\textwidth}{%
            The truth value for this statement is true because the only way to make this
            statement false is to make $x^2 + y^3 > 7$ false and $x+y=5$ true. But
            any values that would make $x^2 + y^3 > 7$ false would make $x+y=5$ false thus 
            making the conditional true.
        }%
    }

    \quad \textbf{(d)} $\exists_x\forall_y \, ( x + y = 5 \rightarrow xy \geq 0)$

    \fbox{%
        \parbox{0.95\textwidth}{%
            The truth value for this statement is false. Consider $y = 5-x$, then $y \in \mathbb{Z}$
            and $x+y = x+ 5 - x=5$ and $xy = 5x - x^2 \geq 0$ which is false for $x > 5$. Thus this
            conditional can be false for some value of x and all values of y.
        }%
    }

    %###################################################################################

    \textbf{Problem 3:}

    See above.

    %###################################################################################

    \textbf{Problem 4:}

    \quad \textbf{(a)} 

    $\boxed{\forall_x \, \exists_y (|x-y| < 1)}$

    \quad \textbf{(b)} 

    \quad \textbf{(c)} 

    \quad \textbf{(d)}

    $\boxed{(x > 1) \land \neg\,\exists_y\,\exists_z\,(x = yz \land y < x \land z < x)}$

    %###################################################################################

    \textbf{Problem 5:}

    \quad \textbf{(a)} 

    $\boxed{\exists_{x\in\mathbb{Z}}\,\forall_{y\in\mathbb{Z}}\,(x^2 \geq y)}$

    \quad \textbf{(b)} 

    $\boxed{\exists_{x\in\mathbb{Z}}\,\forall_{y\in\mathbb{Z}}\,(x < y \land xy\geq 1)}$

    \quad \textbf{(c)} 

    $\boxed{\forall_{x\in\mathbb{R}}\,\exists_{v\in\mathbb{Z}}((x+v=1\land v\leq x) \lor (x+v\neq 1 \land v > x))}$

    %###################################################################################

    \textbf{Problem 6:}

    \fbox{%
        \parbox{0.95\textwidth}{%
        The statement $\exists_x\,P(x)$ says there exists at least one value of x for which
        $P(x)$ is true. Thus the negation of $\exists_x\,P(x)$ says there doesn't exists a value
        of x for which $P(x)$ is true. Which can be also stated as for all values of x, $P(x)$ is false.
        Which is the same as $\forall_x\,(\neg P(x))$. Thus $\neg\exists_x\,P(x)\equiv\forall_x\,(\neg P(x))$.
        }%
    }

    %###################################################################################

    \textbf{Problem 7:}

    \fbox{%
        \parbox{0.95\textwidth}{%
            The statement $\exists_x\,(P(x)\rightarrow Q(x))$ says there exists at least
            one value of x such that $P(x)$ implies $Q(x)$. This statement is further defined
            as, there exists at least one value of x such that either $P(x)$ is false or 
            $Q(x)$ is true. Since the conditional is defined as an or statement, the statement
            could be phrased so that it only relies on $Q(x)$ to make the conditional true. Thus
            $\exists_x\,(P(x)\rightarrow Q(x)) \equiv \forall_x\,P(x) \rightarrow \exists_x\,Q(x)$.
        }%
    }

    %###################################################################################

    \textbf{Problem 8:}

    \fbox{%
        \parbox{0.95\textwidth}{%
            The statement $\exists !xP(x)$ says there exists a unique x such that P(x) is true.
            This statement could be restated as there exists a x and no other
            x exists such that $P(x)$ is true. 

            This can be written as $\exists_x\,P(x) \land$ (no other x exists such that $P(x)$ is true).
            Taking the negation of the right hand side would give the statement,
            $\neg$(other x exists such that $P(x)$ is true). Other x can be replaced with y where x $\neq$ y.

            This leaves us with $\exists_x\, P(x) \land \neg$ (y exists such that $P(y)$ and $x \neq y$).
            This can be further defined using quantifiers.

            $\exists_x\,P(x) \land \neg (\exists_y\,P(y)\land x \neq y) \\
            \exists_x\,P(x) \land (\forall_y\,\neg P(y)\lor x = y)
            $

            We can apply that same process to $\exists !xQ(x)$. Leaving us with:

            $(\exists_x\,P(x) \land (\forall_y\,\neg P(y)\lor x = y)) \rightarrow 
            (\exists_x\,Q(x) \land (\forall_y\,\neg Q(y)\lor x = y))$
        }%
    }

    % %###################################################################################

    % \textbf{Problem 9:}



    % %###################################################################################

    % \textbf{Problem 10:}

\end{document}