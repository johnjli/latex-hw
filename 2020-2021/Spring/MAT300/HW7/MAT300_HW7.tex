\documentclass{article}

\usepackage[a4paper, total={6in, 8in}]{geometry}
\usepackage[utf8]{inputenc}
\usepackage{amsmath}
\usepackage{amssymb}
\usepackage{amsthm}
\usepackage{fancyhdr}
\usepackage{enumitem}
\usepackage[makeroom]{cancel} 

\pagestyle{fancy}
\fancyhf{}
\lhead{John J Li}
\rhead{MAT300 Spring 2021 Homework 7}
\rfoot{\thepage}
\lfoot{April 2021}
\renewcommand{\headrulewidth}{0.4pt}
\renewcommand{\footrulewidth}{0.4pt}

\setlength{\parskip}{1em}
\setlength\parindent{0px}
\title{MAT300 Spring 2021 Homework 7}
\date{\today}
\author{John J Li}

\begin{document}
    \maketitle
    \thispagestyle{empty}

    %################################################################################### 

    \section*{Problem 1}
    
    Check if the following relations $R$ on $\mathbb{Z}$ are functions from $\mathbb{Z}$ to $\mathbb{Z}$.

    (a) $aRb$ if $a\leq b$

    \begin{enumerate}[label=\quad\quad, leftmargin=*]
        \item 
        This relation is not a function. Suppose $a\in\mathbb{Z}$ and $b\in\mathbb{Z}$ and 
        $a\leq b$ then $a\leq b-1$. Since $a\leq b$ and $a\leq b-1$, this is not a function.
    \end{enumerate}

    (b) $aRb$ if $a^2 = b^2$

    \begin{enumerate}[label=\quad\quad, leftmargin=*]
        \item 
        This relation is not a function. Suppose $a=1$ then $b=1,-1$. There is more than 
        one $b\in\mathbb{Z}$ such that $aRb$.
    \end{enumerate}

    (c) $aRb$ if $\{a\} \subseteq \{b\}$

    \begin{enumerate}[label=\quad\quad, leftmargin=*]
        \item 
        This relation is a function. Suppose $\{a\} \subseteq \{b\}$, then $a=b$ since $a,b
        \in\mathbb{Z}$.
    \end{enumerate}


    %###################################################################################

    \section*{Problem 2}

    Let $A,B$ be two disjoint sets and let $f : A \rightarrow A \cup B,\; g : B \rightarrow A \cup B$ be
    functions. Show that $f \cup g$ is a function from $A \cup B$ to $A \cup B$.

    \begin{proof}
        Suppose $A,B$ are disjoint sets and let $f : A \rightarrow A \cup B,\; g : B \rightarrow A \cup B$ be
        functions. Consider $a\in A$, then there exists $x\in A\cup B$ such that $(a,x)\in f$.
        Consider $b\in B$, then there exists $y\in A\cup B$ such that $(b,y)\in g$.
        Consider $f\cup g$, then $(a,x),(b,y)\in f\cup g$ and $a\neq b$ since $A,B$ are 
        disjoint. And $a,b \in A\cup B$ and $x,y\in A\cup B$. Thus $f \cup g$ is a function from $A \cup B$ to $A \cup B$.
    \end{proof}


    %###################################################################################

    \section*{Problem 3}

    Which of the following functions are injective and which are surjective (justify).

    (a) $f:\mathbb{N}\rightarrow\mathbb{N},\; f(n)=\lceil\frac{n+2}{3}\rceil$

    \begin{enumerate}[label=\quad\quad, leftmargin=*]
        \item 
        This function is not injective. Consider $n=2$ then $f(2)=2$, now consider $n=3$ then 
        $f(3) = 2$ so $f(3) = f(2)$ so this function is not injective.
        \item
        This function is not surjective since $\forall_{n\in\mathbb{N}}(f(n) > 0)$.
    \end{enumerate}

    (b) $f:\mathbb{Z}\rightarrow\mathbb{N},\; f(n)=n^2$

    \begin{enumerate}[label=\quad\quad, leftmargin=*]
        \item 
        This function is not injective. Consider $n=-1$ then $f(-1)=1$, now consider $n=1$ then 
        $f(1) = 1$ so $f(-1)=f(1)$, thus this function is not injective.
        \item
        This function is not surjective since $\forall_{n\in\mathbb{Z}}(f(n) \neq 2)$.
    \end{enumerate}
    
    (c) $f:\mathbb{N}\rightarrow\mathbb{N},\; f(n)=|\lfloor\frac{n-1}{5}\rfloor-1|$

    \begin{enumerate}[label=\quad\quad, leftmargin=*]
        \item 
        This function is not injective. Suppose $f(n)=|\lfloor\frac{n-1}{5}\rfloor-1|$ 
        and consider $n=0$, $f(0) = 1$. Now consider $n=1$, $f(1)=1$; and $1,0\in\mathbb{N}$. Since $f(0)=f(1)$
        this function is not injective.
        \item
        This function is surjective. Consider $b\in\mathbb{N}$ and $a=\pm 5b+6$.
        \item
        Case 1: $a=5b+6$.
        \begin{align*}
            f(a)&=|\lfloor\frac{a-1}{5}\rfloor-1| \\
            &=|\lfloor\frac{5b+6-1}{5}\rfloor-1| \\
            &=|\lfloor b+1\rfloor-1| \\
            &=|b+1-1| & \text{(Since $b\in\mathbb{N})$}\\
            &=b 
        \end{align*}
        Case 2: $a= -5b+6$
        \begin{align*}
            f(a)&=|\lfloor\frac{a-1}{5}\rfloor-1| \\
            &=|\lfloor\frac{-5b+6-1}{5}\rfloor-1| \\
            &=|\lfloor -b+1\rfloor-1| \\
            &=|-b+1-1| & \text{(Since $b\in\mathbb{N})$}\\
            &=b 
        \end{align*}
    \end{enumerate}


    %###################################################################################

    \section*{Problem 4}

    Give an example of a function from $\mathbb{Z}$ to $\mathbb{Z}$ which is injective 
    but not surjective and an example of a function which is surjective but not injective.

    Injective but not surjective.

    \begin{enumerate}[label=\quad\quad, leftmargin=*]
        \item $f(n) = 2n$
    \end{enumerate}

    Surjective but not injective.

    \begin{enumerate}[label=\quad\quad, leftmargin=*]
        \item $f(n) = \lfloor \frac{n}{2} \rfloor$
    \end{enumerate}
    %###################################################################################

    \section*{Problem 5}

    Find a bijective function from $\mathbb{Z}$ to the set $\mathbb{Z}\setminus\{-1,0,1\}$
    (Show that your example is indeed a bijection).

    \[
        f(n) = 
        \begin{cases}
            x + 2, & \text{if } n\geq 0 \\
            -x - 1, & \text{if } n < 0
        \end{cases}
    \]

    \begin{enumerate}[label=\quad\quad, leftmargin=*]
        \item 
        This function is injective.
        \begin{enumerate}[label=\quad\quad, leftmargin=*]
            \item
            Case 1: $n\geq 0$. Let $x,y\in\mathbb{N}$ and suppose $f(x)=f(y)$. Then 
            $x+2=y+2$ so $x=y$.
            Case 2: $n<0$. Let $x,y\in\mathbb{N}$ and suppose $f(x)=f(y)$. Then 
            $-x-1 = -y-1$ so $x=y$.
        \end{enumerate}
        \item
        This function is surjective.
        \begin{enumerate}[label=\quad\quad, leftmargin=*]
            \item
            Case 1: $n\geq 0$. Let $y\in\mathbb{N}$ and suppose $x=y-2$. Then 
            $y-2+2=y$.
            Case 2: $n<0$. Let $y\in\mathbb{N}$ and suppose $x=-y-1$. Then 
            $-(-y-1)-1=y+1-1=y$.
        \end{enumerate}
    \end{enumerate}


    %###################################################################################

    \section*{Problem 6}

    Which of the following functions are injective and which are surjective (justify).

    (a) $f:\mathbb{Z}\times\mathbb{Z}\rightarrow\mathbb{Z},\; f((a,b))=a^2+b$

    \begin{enumerate}[label=\quad\quad, leftmargin=*]
        \item 
        This function is not injective. Consider $a=1,b=0$ then $f((1,0))=1$, now consider 
        $a=2,b=-3$ then $f((2,-3))=1$ and $(1,0),(2,-3)\in\mathbb{Z}\times\mathbb{Z}$ 
        so $f((2,-3))=f((1,0))$.
        \item
        This function is surjective. Let $y\in\mathbb{Z}$ consider $x=(0,y)$. 
        Then $f(x)=f((0,y)) = 0+y=y$ and $(0,y)\in\mathbb{N}\times\mathbb{N}$.
    \end{enumerate}

    (b) $f:\mathbb{Z}\rightarrow\mathbb{Z}\times\mathbb{Z},\; f(a) = (\lceil\frac{a}{2}\rceil,\; a)$

    \begin{enumerate}[label=\quad\quad, leftmargin=*]
        \item 
        This function is injective. Suppose $f(x)=f(y)$ then $(\lceil\frac{x}{2}\rceil, x) = (\lceil\frac{x}{2}\rceil, y)$
        Then $\lceil\frac{x}{2}\rceil = \lceil\frac{y}{2}\rceil$, so $x=y$.
        And looking at the second element of the pair, $x=y$.
        \item
        This function is surjective. Let $y\in\mathbb{Z}$ and consider $x=y$ where $x\in\mathbb{Z}$.
        Then $f(x)=f(2y)=(\lceil\frac{y}{2}\rceil,y)$. 
    \end{enumerate}

    (c) $f:\mathbb{Z}\times\mathbb{Z}\rightarrow\mathbb{Z}\times\mathbb{Z},\; f((a,b))=(a+b,2a-b)$

    \begin{enumerate}[label=\quad\quad, leftmargin=*]
        \item 
        This function is injective. Suppose $f(x,y)=f(c,d)$ then $(x+y,2x-y)=(c+d, 2c-d)$. 
        Then $x+y = c+d$ and $2x-y=2c-d$. Adding these two functions gives $3x-y+y=3c-d+d = 
        3x=3c$ so $x=c$. Since $x=c$, $c+y=c+d=y=c-c+d$ so $y=d$. Thus $(x,y)=(c,d)$.
        \item
        This function is not surjective. For $x,y\in\mathbb{Z}$ and suppose $x+y=0$ then 
        $2x-y = 2(-y)-y = -3y \neq 1$. So $(x+y, 2x-y)\neq(0,1)$.
    \end{enumerate}


    %###################################################################################

    \section*{Problem 7}

    Let $f,g$ be functions from $A$ to $A$. Show that:

    (a) If both $f,g$ are injective then so is $f\circ g$.

    \begin{proof}
        Suppose $f,g$ are injective. Let $a_1$ and $a_2$ be arbitrary elements of $A$
        and suppose that $(f\circ g)(a_1)=(f\circ g)(a_2)$. This means $f(g(a_1))=f(g(a_2))$.
        Since $f$ is injective, $g(a_1)=g(a_2)$ and similarly since $g$ is injective, 
        $a_1=a_2$. Thus $f\circ g$ is injective.
    \end{proof}

    (b) If both $f,g$ are surjective then so is $f\circ g$.

    \begin{proof}
        Suppose $f,g$ are surjective and let $a$ be an arbitrary element of $A$. Since 
        $f$ is surjective, there is some $b\in A$ such that $f(b)=a$. Similarly, since 
        $g$ is surjective, there is some $c\in A$ such that $f(c)=b$. Then $(f\circ g)(c) =
        f(g(c)) = f(b) = a$. Thus, $f\circ g$ is surjective.
    \end{proof}

\end{document}