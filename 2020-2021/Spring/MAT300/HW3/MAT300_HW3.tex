\documentclass{article}
\usepackage[utf8]{inputenc}
\usepackage{amsmath}
\usepackage{amssymb}
\usepackage{amsthm}
\usepackage[makeroom]{cancel}

\setlength{\parskip}{1em}
\setlength\parindent{0px}
\title{Homework 3}
\date{\today}
\author{John J Li}

\begin{document}
    \pagenumbering{gobble}
    \maketitle
    \newpage
    \pagenumbering{arabic}

    %###################################################################################

    \textbf{Problem 1:}

    \begin{proof}
        Suppose $0<a<b$. 
        Then multiplying the inequality $a<b$ by the positive number $a$ can
        conclude that $a^2<ab$; similarly, multiplying the inequality $a<b$ by the positive
        number $b$ can conclude that $ab<b^2$. 
        Thus $a^2<ab<b^2$. Then by multiplying the
        new inequality $a^2<ab<b^2$ by the positive number $a$, we can conclude $a^3<a^2b<ab^2$;
        furthermore, by multiplying the
        new inequality $a^2<ab<b^2$ by the positive number $b$, we can conclude $a^2b<ab^2<b^3$.
        So $a^3<a^2b<ab^2<b^3$, thus if $0<a<b$ then $a^3<b^3$.
    \end{proof}

    %###################################################################################

    \textbf{Problem 2:}

    \begin{proof}
        We will prove the contrapositive.
        Suppose if $x=9$ then $x^3-\sqrt{x}\neq 3x^2+sin(x)$. Then $(9)^3-\sqrt{9}\neq 3(9)^2+sin(9)$
        and $726 \neq 243.412$. Therefore if $x^3-\sqrt{x}= 3x^2+sin(x)$ then
        $x\neq 9$.
    \end{proof}

    %###################################################################################

    \textbf{Problem 3:}

    \begin{proof}
        We will prove the contrapositive.
        Suppose n is not even. Then $n = 2k + 1$ where $k \in \mathbb{Z}$.
        Then $3(2k+1)^3 + 6$ and $24k^3 + 36k^2 + 18k^2 + 9$ which can be written as
        $2(12k^3 + 13k^2 + 12k + 4) + 1$ which is not even. Therefore if $3(n)^3 + 6$ then
        $n$ is not even.
    \end{proof}

    %###################################################################################

    \textbf{Problem 4:}

    \begin{proof}
        Let $x$ be arbitary. Suppose $A$ and $B\setminus C$ are disjoint. This means $\forall_x (x \in A \land \neg(x\in B \land x \notin C))$
        and so $\forall_x (x \in A \land x \in B \rightarrow x \in C)$ and this is 
        the logic form of $A\cap B \subseteq C$ since x is arbitary. Therefore if $A$ and $B\setminus C$ are disjoint
        then $A\cap B \subseteq C$.
    \end{proof}

    %###################################################################################

    

\end{document}